Reading list
1) Alone together - Sherry Turkle
2) Medium is the Message - Marshall Mcluhan
3) Amusing Ourselves to Death - Neil Postman (A critic that has heavily been influenced by Mcluhan)
4) Addiction by Design - Natasha Dow Schull
5) The Second Self - Sherry Turkle
6) Life on the Screen: Identity in the Age of Internet - Sherry Turkle
7) Future of the Internet and How to Stop it - Jonathan Zittrain (Will check for predictions)
8) On the Internet - Hubert Dreyfus (One of the most renowned Heidegger scholars, applied some of his theories to technology and internet. 
I want to see if he applied Heidegger's "worldhood" to online worlds, or if he has any material that will help me in this application)
9) Presentation of Self in Everyday Life - Erving Goffman (Referenced by Sherry Turkle as the masterpiece on presentation of the self; may
be helpful to see the limitations of the presentation of self in physical worlds and how they change in online worlds)
10) Status Update - Alice Marwick
11) It's Complicated (Sara Watson, while discussing my thesis topic, recommended these two)
12) Being and Time - Martin Heidegger (The concept of "worldhood", to apply to online worlds)
13) Being and Nothingness - Sartre (Not in relation to Heidegger, not for wide use. I only want to use the concepts of "the gaze" and 
"the Other". Once again, to apply to online worlds)
14) Question Concerning Technology - Martin Heidegger (I cannot remember why I thought it would be useful, but he addresses technology head on,
so I am sure there is something to work with there.)
15) Sickness Unto Death - Kierkegaard (This one might be too far-back, I will see. He has a "hierarchy of selves" that may be useful in
characterizing today.)
16) Understanding Media: The Extention of Man - Marshall Mcluhan (An extended version of Mcluhan's other book; I will look into it if 
I feel that the ideas in the other one are need more exploration.)
17) E Unibus Pluram - David Foster Wallace
18) Twitter and Tear Gas - Zeynep Tufekci
