In the scope of this note, "social media" refers to non-threaded platforms, i.e. , Facebook, Instagram and Twitter.
 
The invisible but omnipresent audience
Social media platforms are designed in a way to encourage the dissipation and visibility of one's material. When the user posts, the tendency (and perhaps the purpose) of the platform is not to keep the post private, but make it available to the public. Even though there are a few options for the user to target a specific group of people in the post, they are not made to be the primary use case scenarios of the platform. For example, Facebook allows for the user to create a group for "best friends" and target a post towards this individual group, but the default is still to make the post public to all of one's connections. Similarly, Instagram allows the user to make his profile 'private' such that he has to approve every user that gets access to the profile, but the default setting is that all profiles are open to the public. Furthermore, even if the user of any platform puts the extra effort to "privatize" the account to some extent, the material users post still reaches a large enough number of people that it is difficult for the user to keep individual track of members of his audience. This has come to such a point that there naturally evolved the concept of a "Finsta", a secondary account that the user only tells her closest connections and feels free to use the platform in a more unfiltered way. Finally, even when all precautions are taken, the user is aware that once his material is "out in the world", there is no way to truly take it back. Once he posts a Tweet, for example, the medium intends for the Tweet to dissipate to a larger audience: A friend shares it and the connection the user has to his audience is diluted, then another viewer shares it, and the Tweet reaches an audience whose members are multiple layers removed from the poster. Even if he deletes a post, he does not have a guarantee that there is no remnant of it out there; no system exists to "forget" all copies of the material one has posted. Even in Snapchat, which originally was set up to be a "no future consequence" app, viewers can take screenshots of the post, and secondary apps are now in use where users can store all of the Snaps they have received. Therefore, the material that a user posts is not only going to be viewed by an unknown group of people, but can continue to be seen at any future point in time. 
 
All of this is to show that deeply ingrained in the social media platforms of today is the idea that the user should expect that her material will reach people who are not just her friends, but connections that are arbitrarily many layers removed, far acquaintances, and even complete strangers. Although this may seem like a natural way for social media to function, there is still an analysis to be done on what life activities on social media take when fundamental to it is the knowledge that the material might reach virtually anyone with an internet connection. Hence, the audience that there is for any user, at least theoretically, is an invisible but omnipresent audience. It is invisible, because the user does not know at the time of posting who might read the message. Furthermore, there is no point in time when the user has a full knowledge of all the individuals who have seen his post. None of Facebook, Instagram or Twitter have a mechanism for showing the specific users who have seen a particular post. On the other hand, the audience is omnipresent because anyone with an internet connection (or with a connection to anyone with an internet connection) is a possible audience member for the post. Also, the audience is not just everywhere, but "every-when". Unlike spoken word, a post does not expire, and even if it gets deleted, there is never a definitive answer as to whether or not the information still exists somewhere (Beyond the database of the platform, of course, where there is a definitive positive answer as to this question). 
 
The invisible but omnipresent audience is a peculiar extension of the Sartrian Other. The Other is the judgmental eye of a particular individual that is observing me (Writing in the first person because it seems important to Sartre to do so). Recognizing the presence of the Other, I can no longer be myself in the way that I am when I am alone. Alone, I only care about the way I am, but observed by the Other, I care about my appearance to the outside world as well. Thus, I lose the freedom that comes with being a subject in a Sartrian sense: "By the mere appearance of the Other, I am put in the position of passing judgment on myself as an object, for it is as an object that I appear to the Other" (Being and Nothingness, 392). It is not important to Sartre whether I am accurate about my belief that the Other is judging me; its "mere appearance" already causes this change on me.
 
Furthermore, The Other is not merely a revelatory power that shows me something about myself; it changes my being. "The Other has not only revealed to me what I was; he has established me in a new type of being which can support new qualifications" (393). The judgment of the Other is important to Sartre not primarily because it shows me a different side of myself, but because it has an effect on my experience of the world. This is not a mere superficial mood change for Sartre, but affects me on an ontological level: on the level of my being: "All of a sudden I am conscious of myself as escaping myself, not in that I am the foundation of my own nothingness but in that I have my foundation outside of myself. I am for myself only as I am a pure reference to the Other" (403). As such, I lose my subjecthood, my freedom to act in a way that I feel that the situation demands of me; "I am no longer the master of the situation" (408).
 
For Sartre, the Other is always a particular individual. But what happens if there is a space (although virtual) that is dedicated to the judgment of the Other; and not the individual Other, but an innumerable crowd of Others. What happens when every individual in this community feels judged before the Others simultaneously. Furthermore, what is the significance of always necessarily being the judge and the judged (the self and the Other) in this community? What if the judgment of the Other is not just a bi-product that I have to handle, but the primary objective of this community? Finally, the gaze of the Other is always described as a solitary experience. What happens when two individuals are interacting on social media, say, as one posts and the other comments on the post? As they both feel the judgment of the Others, do they both become objects? What is their relationship to one another? Is the inter-subjective gaze different?
 
Provisionally, my supposition is that I cannot be free in a radical Sartrian sense when I am in social media. My continuous awareness that I am being judged by the invisible but omnipresent Others rids me of this possibility. Therefore, even when I behave in a way that seems to show no interest in my audience, it is because I would like to appear as such. If the eye of the Other puts my foundation outside of myself, my entire presence in social media must be as an object that exists in reference to my audience. Thus, the users of social media today must always be objects and never free subjects.
 
Further possible directions:
•	The "Inhuman Gaze" that is not a particular and concrete individual is traditionally God. For Early Christians, this is a blessing. For Nietzsche, it is a nightmare. Is the invisible but omnipresent audience the new God? If so, do we cherish it or try to escape it? "There is no single place in space or time from which God gazes upon us. Or rather, he gazes upon us from every place at once – his gaze is omnivoyent." -Kelly, Inhuman Gaze

